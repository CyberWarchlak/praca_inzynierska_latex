% !TeX spellcheck = pl_PL
\chapter{Implementacja}\label{ch:implementation}
\section{Wykorzystywane środowiska i~narzędzia programistyczne}\label{sec:dev-tools}

\lipsum[5]

\section{Zakres implementacji}\label{sec:implementation-scope}

\lipsum[5]

\section{Architektura systemu}\label{sec:system-architecture}

\lipsum[5]
\todo{diagram rozmieszczenia}

\subsection{Architektura backendu}\label{subsec:system-architecture:backend}

\lipsum[5]
\todo{diagram pakietów}

\subsection{Architektura frontendu}\label{subsec:system-architecture:frontend}

\lipsum[5]

\section{Dokumentacja kodu}\label{sec:code-documentation}

Podstawowa dokumentacja kodu została napisana przy użyciu komentarzy w~stylu kompatybilnym z~generatorem dokumentacji JavaDoc\cite{tech:javadoc}.
Przykładowy komentarz przedstawiono na listingu \ref{listing:javadoc}.

\noindent\hspace{.075\textwidth}\begin{minipage}{.85\textwidth}
    \begin{minted}{java}
/**
* Short description of measure.
*/
@NotNull
@Size(min = 1, max = 255)
private String description;
    \end{minted}
    \begin{lstlisting}[caption={Komentarz w~stylu JavaDoc \source{\ownwork}}, label={listing:javadoc}]
\end{lstlisting}
\end{minipage}

\section{Instalacja oprogramowania}\label{sec:software-installation}
\subsection{Wymagania wstępne}\label{subsec:prerequirements}

\lipsum[5]

\subsection{Instalacja}\label{subsec:installation}

\lipsum[5]

\section{Prezentacja aplikacji}\label{sec:app-presentation}

\lipsum[5]

\thispagestyle{normal}
